% \outline{Disinfo project}

Throughout my undergraduate degree, I have worked with Professor Komeili on understanding disinformation and hate speech in social discourse surrounding major Canadian news events. Our goal is to construct a large-scale dataset characterizing the language use and behavior of malicious users, backed by precise annotations as well as natural language explanations. This involves collecting Twitter data and implementing an annotation platform for Amazon Mechanical Turk. The data annotation process is integral to any machine learning project, so I value the experience I've gained in building a stable yet flexible annotation platform, implementing quality control measures, and using the Mechanical Turk service. My senior thesis has taken a similar focus, leveraging natural language instructions from human annotators to develop an interpretable model for identifying hate speech on social media. Allowing the model to identify useful instructions enables a human-in-the-loop process, which could help mitigate bias in hate speech classification \cite{sap-etal-2019-risk}. I have enjoyed how this project has tied together so many of my interests, like interpretability, machine teaching, and online safety. This project is supported by the I-CUREUS grant.